\hypertarget{index_intro_sec}{}\section{Introduction}\label{index_intro_sec}
\hyperlink{class_queued_tracker}{Queued\+Tracker}, or Q\+Trk in short, is an A\+PI that facilitates the 3 dimensional subpixel tracking of a magnetic bead in a Magnetic Tweezers (MT) setup. The code found here generates a general purpose library in the form of a D\+LL that can be used from either .N\+ET applications or Lab\+V\+I\+EW. The \href{https://github.com/NynkeDekkerLab/BeadTracker/tree/master}{\tt Lab\+V\+I\+EW G\+UI and hardware control}, which is not included in this documentation, has been created very specifically for the setups as designed and used in the \href{http://nynkedekkerlab.tudelft.nl/}{\tt Nynke Dekker lab}.\hypertarget{index_hist_sec}{}\section{History}\label{index_hist_sec}
The MT setups are homebuilt devices for biological single molecule measurements. They have evolved over the years and so has the need for the related software. Considering the framerate and number of pixels of state of the art cameras involved in the setups, the requirements with regards to data handling are now very steep. A measurement running for a few hours generates hundreds of gigabytes worth of image data. As such, the need arose to do the image analysis fast, in real-\/time. This software was created to do precisely that. To ensure high speed data analysis, multiple algorithms have been implemented in multithreaded C\+PU and G\+PU (through C\+U\+DA) implementations, with a scheduling shell (\hyperlink{class_queued_c_p_u_tracker}{Queued\+C\+P\+U\+Tracker} and \hyperlink{class_queued_c_u_d_a_tracker}{Queued\+C\+U\+D\+A\+Tracker}) and separate data gathering and saving thread (\hyperlink{class_result_manager}{Result\+Manager}).\hypertarget{index_imple_sec}{}\section{Implementation}\label{index_imple_sec}
The goal is to find a 3 dimensional position of a bead from a microscopic image. A typical image of a single bead is displayed below\+:  To perform the tracking, specific algorithms exist and have been implemented. Currently the available options are\+: \tabulinesep=1mm
\begin{longtabu} spread 0pt [c]{*6{|X[-1]}|}
\hline
\rowcolor{\tableheadbgcolor}{\bf Algorithm }&{\bf Dimensions }&{\bf C\+PU }&{\bf C\+U\+DA }&{\bf Reference }&{\bf Notes }\\\cline{1-6}
\endfirsthead
\hline
\endfoot
\hline
\rowcolor{\tableheadbgcolor}{\bf Algorithm }&{\bf Dimensions }&{\bf C\+PU }&{\bf C\+U\+DA }&{\bf Reference }&{\bf Notes }\\\cline{1-6}
\endhead
Starting point &&\hyperlink{class_queued_c_p_u_tracker_a3c827ffb590b8e80b7e5a585f432ace9}{Queued\+C\+P\+U\+Tracker\+::\+Process\+Job} &\hyperlink{class_queued_c_u_d_a_tracker_a66c822d85cceb3b20793a9f5be76aae0}{Queued\+C\+U\+D\+A\+Tracker\+::\+Execute\+Batch} &&Functions from which the algorithms are called dependent on settings. \\\cline{1-6}
\\\cline{1-6}
Center of Mass (C\+OM) &x,y &\hyperlink{class_c_p_u_tracker_a18c3c6ec23abbbfb018d30b166ed140e}{C\+P\+U\+Tracker\+::\+Compute\+Mean\+And\+C\+OM} &\hyperlink{group__kernels_gaadfa5148ca9461daab04dbf9e0394791}{Bg\+Corrected\+C\+OM} &&Always executed for first guess. \\\cline{1-6}
1D Cross Correlation (X\+Cor1D) &x,y &\hyperlink{class_c_p_u_tracker_aeb547c7da30e1621b2869634182c0900}{C\+P\+U\+Tracker\+::\+Compute\+X\+Cor\+Interpolated} &Not implemented &&\\\cline{1-6}
Quadrant Interpolation (QI) &x,y &\hyperlink{class_c_p_u_tracker_ab856aa12313a07c083f2e193180fd5b6}{C\+P\+U\+Tracker\+::\+Compute\+QI} &\hyperlink{class_q_i}{QI}, \hyperlink{class_q_i_a084274d952e1430627110818f398a3d4}{Q\+I\+::\+Execute} &&Recommended algorithm. Optimized for speed and accuracy. \\\cline{1-6}
2D Gaussian fit &x,y &\hyperlink{class_c_p_u_tracker_a99567e8137e89ff9ee084558e9f26344}{C\+P\+U\+Tracker\+::\+Compute2\+D\+Gaussian\+M\+LE} &\hyperlink{group__kernels_gaf3546eed501c5227c765beb290ed2549}{G2\+M\+L\+E\+\_\+\+Compute} &&\\\cline{1-6}
Lookup table (L\+UT) &z &\hyperlink{class_c_p_u_tracker_a425db1a2a66d9635d2b4b565ecfdc3b0}{C\+P\+U\+Tracker\+::\+Compute\+Radial\+Profile} ~\newline
\hyperlink{class_c_p_u_tracker_a605758e0bf1f897f86f38b65e99e320b}{C\+P\+U\+Tracker\+::\+L\+U\+T\+Profile\+Compare} &\hyperlink{group__kernels_gadf9148f47982d2685fa156a957fc21c2}{Z\+L\+U\+T\+\_\+\+Radial\+Profile\+Kernel} ~\newline
\hyperlink{group__kernels_gad0ba2ca03fcfe17bda05c872842afaae}{Z\+L\+U\+T\+\_\+\+ComputeZ} &&Only available method for z localization. \\\cline{1-6}
\end{longtabu}
\hypertarget{index_soft_sec}{}\section{Required software}\label{index_soft_sec}
To be able to compile the D\+L\+Ls, you need\+:
\begin{DoxyItemize}
\item Visual Studio (2010)
\item C\+U\+DA (6.\+5)
\end{DoxyItemize}

To be able to use the C\+U\+DA D\+L\+Ls, the cudart32\+\_\+65.\+dll and cufft32\+\_\+65.\+dll (or 64 bit versions if compiled with that) need to be known and accessible by the system, i.\+e. they need to be in the same folder. We are currently limited to C\+U\+DA 6.\+5 at the highest due to the fact that we have a 32 bit Lab\+V\+I\+EW version on setups and 32 bit cu\+F\+FT has been deprecated in newer C\+U\+DA versions.\hypertarget{index_cred_sec}{}\section{Credits}\label{index_cred_sec}
Original work by Jelmer Cnossen. Maintenance, testing, documentation and improvements by Jordi Wassenburg. 